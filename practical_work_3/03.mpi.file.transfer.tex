\documentclass{article}
\usepackage[utf8]{inputenc}
\usepackage{geometry}
\usepackage{listings}
\usepackage{float}
\geometry{a4paper, margin=1in}

\title{Practical Work 3: MPI File Transfer}
\author{Vo Truong Giang - BI12-130}
\date{\today}

\begin{document}
\maketitle

\section{Introduction}
In this practical work, I implemented a file transfer system using the Message Passing Interface (MPI). Unlike the previous Client-Server models (TCP/RPC), MPI allows processes to communicate within a parallel computing environment.

\section{MPI Implementation Choice}
I chose \textbf{OpenMPI} because it is the standard, open-source implementation of MPI on Linux/Ubuntu systems. It provides robust tools like \texttt{mpicc} for compilation and \texttt{mpirun} for execution.

\section{System Design}
The system runs as a single executable but behaves differently based on the process rank:
\begin{itemize}
    \item \textbf{Rank 0 (Receiver):} Acts as the server. It waits for messages with \texttt{TAG\_FILENAME} then \texttt{TAG\_CONTENT}.
    \item \textbf{Rank 1 (Sender):} Acts as the client. It reads the file and sends data to Rank 0.
\end{itemize}

\begin{figure}[H]
\centering
\begin{verbatim}
RANK 1 (Sender)                               RANK 0 (Receiver)
   |                                                |
   |--- MPI_Send(Filename, TAG_FILENAME) ---------->| MPI_Recv
   |                                                |
   |--- MPI_Send(Data Chunk, TAG_CONTENT) --------->| MPI_Recv (Loop)
   |--- MPI_Send(Data Chunk, TAG_CONTENT) --------->| MPI_Recv
   |                  ...                           |
   |--- MPI_Send(0 bytes, TAG_CONTENT) ------------>| MPI_Recv (Stop)
   |                                                |
\end{verbatim}
\caption{MPI Point-to-Point Communication Design}
\end{figure}

\section{Implementation Details}
\begin{lstlisting}[language=C, basicstyle=\small, frame=single]
// Main logic based on Rank
MPI_Comm_rank(MPI_COMM_WORLD, &world_rank);

if (world_rank == 0) {
    run_server(); // Uses MPI_Recv loop
} else if (world_rank == 1) {
    run_client(argv[1]); // Uses MPI_Send loop
}
\end{lstlisting}

\section{Conclusion}
The MPI implementation successfully transfers files between two processes using blocking point-to-point communication routines.

\end{document}