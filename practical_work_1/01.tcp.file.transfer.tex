\documentclass{article}
\usepackage[utf8]{inputenc}
\usepackage{geometry}
\usepackage{listings}
\usepackage{float}


\geometry{a4paper, margin=1in}

\title{Practical Work 1: TCP File Transfer}
\author{Võ Trường Giang - BI12-130} 
\date{\today}

\begin{document}

\maketitle

\section{Introduction}
In this practical work, I developed a Command Line Interface (CLI) application to transfer files between a Client and a Server using C sockets over TCP/IP. The system ensures reliability and follows a specific application-level protocol.

\section{Protocol Design}
Below is the text-based diagram illustrating the interaction between the Client and the Server. The protocol uses a handshake mechanism ("OK" signal) to ensure the server is ready before the file content is streamed.

\begin{figure}[H]
\centering
\begin{verbatim}
CLIENT                                                  SERVER

connect ---------------------------------------------> accept
   |                                                      |
send(filename) --------------------------------------> recv(filename)
   |                                                      |
   |                                                   fopen("received_...")
   |                                                      |
   |<------------------------------------------ send("OK")|
recv("OK")                                                |
   |                                                      |
send(file data chunks) ------------------------------> recv(buffer)
   |            (Loop until End of File)                  |--> fwrite(buffer)
   |                                                      |
close -----------------------------------------------> close
\end{verbatim}
\caption{Protocol Sequence Diagram}
\end{figure}

\section{System Organization}
The system architecture follows the Client-Server model:
\begin{itemize}
    \item \textbf{Server:} Binds to port 8080 and listens for connections. It saves received files with a "received\_" prefix.
    \item \textbf{Client:} Connects to the Server's IP (127.0.0.1) and streams the file data.
\end{itemize}

\section{Implementation Details}
Below are the core code snippets demonstrating the transfer logic.

\subsection{Server Side (Receiver)}
\begin{lstlisting}[language=C, basicstyle=\small, frame=single]
// 1. Receive Filename
read(new_socket, buffer, BUFFER_SIZE);
sprintf(filename, "received_%s", buffer);

// 2. Send Acknowledgement
send(new_socket, "OK", 2, 0);

// 3. Receive Data Loop
FILE *fp = fopen(filename, "wb");
while ((valread = recv(new_socket, buffer, BUFFER_SIZE, 0)) > 0) {
    fwrite(buffer, 1, valread, fp);
}
fclose(fp);
\end{lstlisting}

\subsection{Client Side (Sender)}
\begin{lstlisting}[language=C, basicstyle=\small, frame=single]
// 1. Send Filename
send(sock, filename, strlen(filename), 0);

// 2. Wait for Server Readiness
read(sock, buffer, BUFFER_SIZE);
if (strcmp(buffer, "OK") == 0) {
    // 3. Send Data Loop
    while ((bytes_read = fread(buffer, 1, BUFFER_SIZE, fp)) > 0) {
        send(sock, buffer, bytes_read, 0);
    }
}
\end{lstlisting}

\section{Conclusion}
The implementation successfully transfers both text and binary files (like images). The use of the "OK" acknowledgment ensures the Server is ready before data is sent.

\end{document}