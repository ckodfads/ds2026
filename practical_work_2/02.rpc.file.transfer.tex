\documentclass{article}
\usepackage[utf8]{inputenc}
\usepackage{geometry}
\usepackage{listings}
\usepackage{float}
\geometry{a4paper, margin=1in}

\title{Practical Work 2: RPC File Transfer}
\author{Vo Truong Giang - BI12-130}
\date{\today}

\begin{document}
\maketitle

\section{Introduction}
In this practical work, I upgraded the file transfer system from TCP sockets to Remote Procedure Calls (RPC) using the ONC RPC (Sun RPC) standard.

\section{RPC Service Design}
Unlike the streaming approach of TCP sockets, RPC allows the client to invoke a function on the server directly.
\begin{itemize}
    \item \textbf{Input:} A structure containing the filename (string) and file content (opaque data).
    \item \textbf{Output:} An integer status code (1 for success, 0 for failure).
    \item \textbf{Tool used:} \texttt{rpcgen} to generate stubs and XDR filters.
\end{itemize}

\section{System Organization}
The system consists of:
\begin{itemize}
    \item \textbf{transfer.x:} The Interface Definition Language (IDL) file.
    \item \textbf{Server:} Implements the \texttt{upload\_file\_1\_svc} function to write data to disk.
    \item \textbf{Client:} Reads the file into memory and calls \texttt{upload\_file\_1}.
\end{itemize}

\section{Implementation}
\subsection{Protocol Definition (.x file)}
\begin{lstlisting}[basicstyle=\small, frame=single]
struct file_data {
    string name<256>;
    opaque data<>;
    int bytes_sent;
};
program FILE_TRANSFER_PROG {
    version V1 {
        int UPLOAD_FILE(file_data) = 1;
    } = 1;
} = 0x31230000;
\end{lstlisting}

\section{Conclusion}
The RPC implementation abstracts the networking details, allowing the developer to focus on the logic of function calls. The system successfully transfers files using the defined RPC interface.

\end{document}